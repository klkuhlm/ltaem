\documentclass{article}
\usepackage{amsmath}
\usepackage{fullpage}
\usepackage{natbib}

\begin{document}
\section{Input File}
\label{sec:input}
The input format for LT-AEM attempts to be somewhat flexible,  but it is fairly rigid because parsing strings in Fortran is awful.  

The following simple input file to compute contours of results on a 40$\times$40
grid, at 4 different times.

\begin{verbatim}
T  F  T  T  1  
contours.out  dump  results.elem  results.geom  
1.0D0   2.0D-2   1.0D-5   
0   1.0D0   1.0D-4   1.0D0   
False 1.5D-1  2.0D0  1.0  
False 1.0D-5  2.0D0    
40  40  4  
OBS_NAME_NOT_USED  
LINVEC -2.5 2.5 
LINVEC -2.5 2.5 
LOGVEC -3.0 0.0 4
1.0D-6  1.0D-8  10  
1  circles.in   
0  20 NOT_USED  
NOT_USED  
\end{verbatim}

\begin{verbatim}
CALC? PARTICLE? CONTOUR? TDERIV?  OUTPUT_TYPE  OUTPUT_FN  DUMP_FN  ELEM_ORG_FN  ELEM_GEOM_FN  
POROSITY  K  Ss   
LEAK_FLAG  AQT_K AQT_Ss  AQT_b  
Sy  Ks  False  1.0  
1.0D-5  2.0D0  False  
40  40  4  
OBS_LOC_NAMES
LINVEC -2.5 2.5 
LINVEC -2.5 2.5 
LOGVEC -3.0 0.0 4
1.0D-6  1.0D-8  10  
1  circles.in   
0  ELLIPSE_MS not_used  
not_used  
\end{verbatim}


\section{kappa}
\label{sec:kappa}

The LT-AEM is able to compute the solution to different PDEs that can
be cast into a similar form as the Laplace-transformed diffusion
equation.  Of most concern for groundwater is the inclusion of various
homogeneous source terms, like the effects of leaky aquitards, aquifer
unconfinedness, and dual porosity.

The original diffusion equation is defined in terms of 
\begin{equation}
  \label{eq:1}
  q^2 = \frac{p}{\alpha}
\end{equation}
where $\alpha=K/S_s$ is aquifer diffusivity.  

\subsection{Leaky}
In the case of leakiness, the three cases considerd are:
\begin{enumerate}
\item[I] no-drawdown condition at opposite side of finite aquitard
\item[II] no-flow condition at opposite side of finite aquitard
\item[III] aquitard thickness is semi-infinite
\end{enumerate}
Two leaky aquitards can be included, simply by including the two types
of terms required (and keeping track of both sets of parameters).
These three source terms are:
\begin{align}
  q_{\mathrm{I}}^2 &= \frac{p}{\alpha} + \frac{p K_2}{\alpha_2 Kb} \frac{1 + \exp\left(-2 \frac{p b_2}{\alpha_2}\right)}{1 - \exp\left(-2 \frac{p b_2}{\alpha_2}\right)} \\
  q_{\mathrm{II}}^2 &= \frac{p}{\alpha} + \frac{p K_2}{\alpha_2 Kb} \frac{1 - \exp\left(-2 \frac{p b_2}{\alpha_2}\right)}{1 + \exp\left(-2 \frac{p b_2}{\alpha_2}\right)} \\
  q_{\mathrm{III}}^2 &= \frac{p}{\alpha} + \frac{p K_2}{\alpha_2 Kb} 
\end{align}
where a subscript $2$ indicates a property of the aquitard.  These
solutions are taken directly from the modified theory of leaky
aquifers \cite{hantush1960modification}.

\subsection{Unconfined}
Unconfined behavior analogous to \cite{neuman1972theory}, can be
obtained by computing an integrated response (since Neuman's solution
has an explict $z$-coordinate), or by using the
\cite{boulton1954drawdown} solution that has no $z$ dependence.

\subsection{Dual Porosity}
Dual-porosity response assumes there are two overlapping domains, the
fractures and the matrix, each most generally with a diffusion-type
solution \cite{dougherty1984flow}.  Under the assumtion that $K_f \gg
K_m$ the 3D diffusion in the matrix reduces to just
\begin{equation}
  \label{eq:2}
  \frac{\partial h_m}{\partial t} = \lambda \left( h_m - h_f\right)
\end{equation}

\bibliographystyle{plainnat}
\bibliography{design}

\end{document}
