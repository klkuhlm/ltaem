\documentclass[12pt,letterpaper]{article}
\usepackage{amsmath}
\usepackage{fullpage}
\usepackage{natbib}
\usepackage{colortbl,multirow}
\usepackage{hyperref}

\begin{document}
\section{Input File}
\label{sec:input}
The input format for LT-AEM attempts to be somewhat flexible,  but it is fairly rigid because parsing strings in Fortran is awful.  

The following simple input file to compute contours of results on a 40$\times$40
grid, at 4 different times.

\begin{verbatim}
T  F  T  T  1  
contours.out  dump  results.elem  results.geom  
1.0D0   2.0D-2   1.0D-5   
0   1.0D0   1.0D-4   1.0D0   
False 1.5D-1  2.0D0  1.0  
False 1.0D-5  2.0D0    
40  40  4  
LOC_1 | LOC_2
LINVEC -2.5 2.5 
LINVEC -2.5 2.5 
LOGVEC -3.0 0.0
1.0D-6  1.0D-8  10  
1  20 circles.in   
0  20 NOT_USED  
NOT_USED  
\end{verbatim}

\begin{verbatim}
CALC? PARTICLE? CONTOUR? TDERIV? OUTPUT_TYPE OUTPUT_FN DUMP_FN ELEM_ORG_FN ELEM_GEOM_FN  
POROSITY  K  Ss   
LEAK_FLAG  AQT_K AQT_Ss  AQT_b  
UNCONFINED_FLAG?, Sy  Ks  AQF_b
DUALPORO_FLAG?, MATRIX_Ss, LAMBDA
NUM_X, NUM_Y, NUM_T
OBS_LOC_NAMES
XVALS
YVALS
TVALS
DKS_ALPHA  DKS_TOL  DKS_M  
NUM_CIRCLES  CIRCLE_INPUT_FN   
NUM_ELLIPSES BG_MS ELLIPSE_INPUT_FN
PARTICLE_INPUT_FN
\end{verbatim}

TODO: explain input format/style

\section{Problem setup and solution}
\label{sec:problem-setup}
The LT-AEM problem is posed in a direct fashion, with individual
elements representing ``sub-blocks'' of the larger system matrix,
which is solved via the ZGELS (double-precision complex (Z) GEneral
Least Squares) LAPACK routine. The sub-blocks are built up in the
\texttt{solution.f90} routine, through calls to the
\texttt{circle\_match\_self}, \texttt{circle\_match\_other},
\texttt{ellipse\_match\_self}, and \texttt{ellipse\_match\_other}
routines.  Each time these routines are called, they return a data
structure consisting of a LHS and RHS matrix.  The solution routine
then assembles the global LHS and RHS matrices from these sub
matrices, and calls ZGELS.

   \begin{equation}
  \label{eq:fullmatrix}
  \left[ \begin{array}{cc|cc|cc}
      \cellcolor[gray]{0.9}\frac{\bar{\Phi}^{1+}}{K_0} & 
      \cellcolor[gray]{0.9}-\frac{\bar{\Phi}^{1-}}{K_1}  
      & \frac{\bar{\Phi}^{2+}}{K_0}  &0&0&0\\
      \cellcolor[gray]{0.9} \mathbf{n}_1 \cdot \bar{\mathbf{q}}^{1+} & 
      \cellcolor[gray]{0.9} -\mathbf{n}_1 \cdot \bar{\mathbf{q}}^{1-} 
      & - \mathbf{n}_1 \cdot \bar{\mathbf{q}}^{2+} &0&0&0\\
      \hline
      \frac{\bar{\Phi}^{1+}}{K_0}  &0& \cellcolor[gray]{0.9} \frac{\bar{\Phi}^{2+}}{K_0} 
      & \cellcolor[gray]{0.9}-\frac{\bar{\Phi}^{2-}}{K_2}  & 
      -\frac{\bar{\Phi}^{3+}}{K_2}  & 0 \\
       \mathbf{n}_2 \cdot \bar{\mathbf{q}}^{1+}  & 0 & 
       \cellcolor[gray]{0.9} \mathbf{n}_2 \cdot \bar{\mathbf{q}}^{2+}
      & \cellcolor[gray]{0.9} -\mathbf{n}_2 \cdot \bar{\mathbf{q}}^{2-} &
      - \mathbf{n}_2 \cdot \bar{\mathbf{q}}^{3+} & 0 \\
      \hline
      0&0&0& \frac{\bar{\Phi}^{2-}}{K_2} & \cellcolor[gray]{0.9}\frac{\bar{\Phi}^{3+}}{K_2} 
      & \cellcolor[gray]{0.9}-\frac{\bar{\Phi}^{3-}}{K_3} \\
      0&0&0& \mathbf{n}_3 \cdot \bar{\mathbf{q}}^{2-} & 
      \cellcolor[gray]{0.9} \mathbf{n}_3 \cdot \bar{\mathbf{q}}^{3+} &
      \cellcolor[gray]{0.9}-\mathbf{n}_3 \cdot \bar{\mathbf{q}}^{3-}
    \end{array} \right] 
\end{equation}


\section{Problem Types}
\label{sec:problem-types}
The LT-AEM program is designed to solve several related problems,
designated with flags in the input file.  Generally, the geometry of
the elements is specified the same between all cases, it is just the
locations, times, types and types of calculated output that change.

LT-AEM can compute:
\begin{itemize}
\item particle tracking forward or backward from a set of initial
  locations and times;
\item solution at a grid of space locations, for a set of times (e.g.,
  for creating contour maps); and
\item solution at a set of named space locations, for a set of times
  (i.e., drawdown time series).
\end{itemize}

\subsection{Particle Tracking}
When performing particle tracking, LT-AEM can compute two different
styles of particle tracking
\begin{itemize}
\item forward or reverse \textbf{pathlines}, either at fixed time steps or
  using adaptive integration (i.e., tracing the path a particle of
  water would take); and
\item forward or reverse \textbf{streaklines}, computing location of a set of
  particles at fixed time intervals (i.e., showing future positions of
  a set of particles released at the same time, like a cloud of
  smoke).
\end{itemize}

The input format for the particle tracking file is given below:
\begin{verbatim}
NUM_PART STREAK_SKIP
np*FWD_TRACK?
np*RKM_TOLERANCE
np*RKM_MAX_DT
np*RKM_MIN_DT
np*INITIAL_DT
np*X0
np*Y0
np*T0
np*T1
np*INT_METHOD_FLAG
np*BEGIN_INSIDE_ELEMENT?
\end{verbatim}
where ``\texttt{np*}'' means the entry is repeated for each of the $np$
particles.  LT-AEM recognizes the following entry shortcut: if there
are not enough entries on the line for all the particles, the first
entry will be repeated for all particles.

\subsubsection{Pathlines}
Pathline computation takes particle start locations and start times as
input, with a flag indicating whether particles are forward or
backward tracking, a flag indicating whether the particle starts in
the background or an element, and a choice of integration method.

The available methods for particle integration are
\begin{itemize}
\item forward Euler (non-adaptive first order)
\item Runge-Kutta (non-adaptive fourth order) consisting of:
  \begin{enumerate}
  \item forward half-step Euler (predictor)
  \item backward half-step Euler (corrector)
  \item full-step midpoint rule (predictor)
  \item full-step Simpson's rule (corrector)
  \end{enumerate}
\item Runge-Kutta-Merson (adaptive fourth order) consisting of:
  \begin{enumerate}
  \item forward third-step Euler (predictor)
  \item third-step trapezoid rule (corrector)
  \item half-step Adams-Bashforth (predictor)
  \item full-step Adams-Bashforth (predictor)
  \item full-step Simpson's rule (corrector)
  \end{enumerate}
  with an adaptive step size based on relative error of the last two
  fourth-order steps.
\item semi-analytical Laplace-space root finding solution, using
  Muller's algorithm for complex roots.
\end{itemize}

\subsubsection{Streaklines}
Streaklines are computed from the output of a non-adaptive particle
tracking algorithm (not Runge-Kutta-Merson).  The ``streakSkip''
parameter indicates how many equal-sized steps should be skipped
between calculation of streak locations.  Aside from this it is just
like the other non-adaptive particle tracking.

\section{kappa}
\label{sec:kappa}

The LT-AEM is able to compute the solution to different PDEs that can
be cast into a similar form as the Laplace-transformed diffusion
equation.  Of most concern for groundwater is the inclusion of various
homogeneous source terms, like the effects of leaky aquitards, aquifer
unconfinedness, and dual porosity.

The original diffusion equation is defined in terms of 
\begin{equation}
  \label{eq:1}
  q^2 = \frac{p}{\alpha}
\end{equation}
where $\alpha=K/S_s$ is aquifer diffusivity.  

\subsection{Leaky}
In the case of leakiness, the three cases considerd are:
\begin{enumerate}
\item[I] no-drawdown condition at opposite side of finite aquitard
\item[II] no-flow condition at opposite side of finite aquitard
\item[III] aquitard thickness is semi-infinite
\end{enumerate}
Two leaky aquitards can be included, simply by including the two types
of terms required (and keeping track of both sets of parameters).
These three source terms are:
\begin{align}
  q_{\mathrm{I}}^2 &= \frac{p}{\alpha} + \frac{p K_2}{\alpha_2 Kb} \frac{1 + \exp\left(-2 \frac{p b_2}{\alpha_2}\right)}{1 - \exp\left(-2 \frac{p b_2}{\alpha_2}\right)} \\
  q_{\mathrm{II}}^2 &= \frac{p}{\alpha} + \frac{p K_2}{\alpha_2 Kb} \frac{1 - \exp\left(-2 \frac{p b_2}{\alpha_2}\right)}{1 + \exp\left(-2 \frac{p b_2}{\alpha_2}\right)} \\
  q_{\mathrm{III}}^2 &= \frac{p}{\alpha} + \frac{p K_2}{\alpha_2 Kb} 
\end{align}
where a subscript $2$ indicates a property of the aquitard.  These
solutions are taken directly from the modified theory of leaky
aquifers \cite{hantush1960modification}.

\subsection{Unconfined}
Unconfined behavior analogous to \cite{neuman1972theory}, can be
obtained by computing an integrated response (since Neuman's solution
has an explict $z$-coordinate), or by using the
\cite{boulton1954drawdown} solution that has no $z$ dependence.

\subsection{Dual Porosity}
Dual-porosity response assumes there are two overlapping domains, the
fractures and the matrix, each most generally with a diffusion-type
solution \cite{dougherty1984flow}.  Under the assumtion that $K_f \gg
K_m$ the 3D diffusion in the matrix reduces to just
\begin{equation}
  \label{eq:2}
  \frac{\partial h_m}{\partial t} = \lambda \left( h_m - h_f\right)
\end{equation}

\bibliographystyle{plainnat}
\bibliography{design}

\end{document}

%%% Local Variables:
%%% mode: latex
%%% TeX-master: t
%%% End:
